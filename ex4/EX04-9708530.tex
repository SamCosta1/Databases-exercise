
\documentclass{article}
\usepackage{graphicx}
\usepackage{listings}
\usepackage[a4paper]{geometry}
\graphicspath{ {.} }
\title{COMP23111 2016 - 2017\ EX4}
\author{Sameul Da Costa - 9708530}
\begin{document}
\maketitle

\section{Comments}
I have made the assumption that columns should be formatted in the individual
exercises and that the formatting should be removed afterwords. This has lead to some repitition
but overall I believe it splits the exercises up more effectively.

Furthermore certain things in part 1g \emph{should} throw errors, please see comments for details
\section{Code}
\begin{lstlisting}
SQL> -- 1a ------
SQL> CREATE OR REPLACE VIEW customers_with_shopping_carts
  2  AS
  3  	 SELECT
  4  	 DISTINCT    firstName,
  5  		     lastName
  6  	 FROM	     customerInfo
  7  	 INNER JOIN  orderCartInfo
  8  	 ON	     orderCartInfo.customerID=customerInfo.loginName;

View created.

SQL> 
SQL> SELECT * FROM customers_with_shopping_carts;

FIRSTNAME            LASTNAME                                                   
-------------------- --------------------                                       
Ophelia              Elsinore                                                   
Mary                 Folio                                                      
Percy                Byron                                                      

SQL> 
SQL> -- 1b ------
SQL> 
SQL> CREATE OR REPLACE VIEW items_to_reorder
  2  AS
  3  	 SELECT  code,
  4  		 inventoryItem.itemNum,
  5  		 belongsTo AS categoryID,
  6  		 qtyInstock AS quantityInStock
  7  	 FROM	 inventoryItem
  8  	 INNER JOIN  itemType
  9  	 ON	 inventoryItem.itemNum = itemType.itemNum
 10  	 WHERE	 qtyInstock < 25;

View created.

SQL> 
SQL> COLUMN CODE FORMAT A5
SQL> SELECT * FROM items_to_reorder;

CODE  ITEMNUM    CATEGORYID QUANTITYINSTOCK                                     
----- ---------- ---------- ---------------                                     
hbk   C2         H                       20                                     
ebk   A0         P                       15                                     

SQL> CLEAR COLUMNS
SQL> 
SQL> -- 1c ------
SQL> CREATE OR REPLACE VIEW order_details
  2  AS
  3  	 SELECT      loginName,
  4  		     firstName,
  5  		     lastName,
  6  		     orderCartInfo.orderCartId,
  7  		     SUM(qtyOrdered * orderPrice)  AS TotalPrice
  8  	 FROM	     customerInfo
  9  	 INNER JOIN  orderCartInfo
 10  	 ON	     orderCartInfo.customerID = customerInfo.loginName
 11  	 INNER JOIN  lineItems
 12  	 ON	     lineItems.orderCartId=orderCartInfo.orderCartId
 13  	 GROUP BY    orderCartInfo.orderCartId,
 14  		     loginName,
 15  		     firstName,
 16  		     lastName;

View created.

SQL> 
SQL> COLUMN FIRSTNAME FORMAT A10
SQL> COLUMN LASTNAME FORMAT A10
SQL> COLUMN ORDERCARTID FORMAT A12
SQL> SELECT * FROM order_details;

LOGINNAME  FIRSTNAME  LASTNAME   ORDERCARTID  TOTALPRICE                        
---------- ---------- ---------- ------------ ----------                        
shkeandco  Mary       Folio      5                  2.99                        
astrab     Percy      Byron      4                 14.97                        
shkeandco  Mary       Folio      1                  8.97                        
dramab     Ophelia    Elsinore   2                 11.97                        
astrab     Percy      Byron      8                  7.98                        
astrab     Percy      Byron      3                 73.83                        
shkeandco  Mary       Folio      6                 13.98                        
dramab     Ophelia    Elsinore   7                  1.99                        

8 rows selected.

SQL> CLEAR COLUMNS
SQL> 
SQL> -- 1d ------
SQL> COLUMN FIRSTNAME FORMAT A10
SQL> COLUMN LASTNAME FORMAT A10
SQL> CREATE OR REPLACE VIEW customer_order_details
  2  AS
  3  	 SELECT      loginName,
  4  		     firstName,
  5  		     lastName,
  6  		     SUM(totalPrice) AS totalPrice
  7  	 FROM	     order_details
  8  	 GROUP BY    loginName,
  9  		     firstName,
 10  		     lastName;

View created.

SQL> 
SQL> SELECT * FROM customer_order_details;

LOGINNAME  FIRSTNAME  LASTNAME   TOTALPRICE                                     
---------- ---------- ---------- ----------                                     
shkeandco  Mary       Folio           25.94                                     
astrab     Percy      Byron           96.78                                     
dramab     Ophelia    Elsinore        13.96                                     

SQL> CLEAR COLUMNS
SQL> 
SQL> -- 1e ------
SQL> 
SQL> CREATE OR REPLACE VIEW carts_per_customer
  2  AS
  3  	 SELECT      loginName,
  4  		     COUNT(orderCartId) AS numOrders
  5  	 FROM	     customerInfo
  6  	 INNER JOIN  orderCartInfo
  7  	 ON	     customerInfo.loginName = orderCartInfo.customerId
  8  	 GROUP BY    loginName;

View created.

SQL> 
SQL> SELECT  loginName,
  2  	     CASE
  3  		 WHEN	 numOrders <= 2
  4  		 THEN	 'BR-1 satisfied'
  5  		 ELSE	 'BR-1 violated'
  6  	     END AS BR_1
  7  FROM    carts_per_customer;

LOGINNAME  BR_1                                                                 
---------- --------------                                                       
dramab     BR-1 satisfied                                                       
shkeandco  BR-1 violated                                                        
astrab     BR-1 violated                                                        

SQL> 
SQL> -- 1f ------
SQL> 
SQL> SELECT  itemNum,
  2  	     itemColor,
  3  	     itemSize
  4  FROM
  5  	 (
  6  	     SELECT  itemNum,
  7  	     itemColor,
  8  	     itemSize,
  9  	     CASE
 10  	     WHEN    COUNT_COL_SIZE <= 1
 11  	     THEN    'BR-2 satisfied'
 12  	     ELSE    'BR-2 violated'
 13  	     END AS BR_2
 14  	     FROM
 15  	     (
 16  		 SELECT  itemNum,
 17  		 itemColor,
 18  		 itemSize,
 19  		 COUNT(itemColor) AS COUNT_COL_SIZE
 20  		 FROM	 inventoryItem
 21  		 GROUP BY itemNum, itemColor, itemSize
 22  	     )
 23  	 )
 24  WHERE BR_2='BR-2 violated';

ITEMNUM    ITEMCOLOR         ITEMSIZE                                           
---------- --------------- ----------                                           
C2         black                    3                                           
A0         green                    4                                           

SQL> 
SQL> -- 1g ------
SQL> 
SQL> SHOW ERRORS TRIGGER check_item_value
No errors.
SQL> SET PAGES 999;
SQL> SET LONG 999999;
SQL> COLUMN "Picture" FORMAT A7;
SQL> 
SQL> CREATE OR REPLACE TRIGGER check_item_value
  2  	 AFTER INSERT OR UPDATE ON itemType
  3  	 DECLARE
  4  	     min_price	     float DEFAULT 0;
  5  	     max_price	     float DEFAULT 0;
  6  	 BEGIN
  7  	     SELECT  MIN(price) INTO min_price
  8  	     FROM    itemType;
  9  
 10  	     SELECT  MAX(price) INTO max_price
 11  	     FROM    itemType;
 12  
 13  	     IF max_price > min_price * 4
 14  	     THEN
 15  		 RAISE_APPLICATION_ERROR(num => -20000,
 16  					 msg => 'Price too high');
 17  	     END IF;
 18  	 END check_item_value;
 19  /

Trigger created.

SQL> 
SQL> -- Initial state of table
SQL> SELECT * FROM itemType;

ITEMNUM    NAME                 PICTURE      PRICE BELONGSTO                    
---------- -------------------- ------- ---------- ----------                   
A0         The Anarchy of Mask  ***          10.99 P                            
A1         The Butler Did       ---          11.99 C                            
A2         The Abolished Man    ===          15.99 SF                           
A3         Lyrical Bullets      +-+-         20.99 P                            
B1         The Postlude         =\=\=        34.99 P                            
C1         The August of Guns   **--         10.99 H                            
C2         The Expectant Mirror ^-^          12.99 H                            

7 rows selected.

SQL> 
SQL> -- Valid insert
SQL> INSERT into itemType VALUES ('D4', 'The 5th thing', '++-+-', 10.88, 'P');

1 row created.

SQL> 
SQL> -- Invalid inserts
SQL> INSERT into itemType VALUES ('C3', 'The thing', '++-+-', 500.99, 'P');
INSERT into itemType VALUES ('C3', 'The thing', '++-+-', 500.99, 'P')
            *
ERROR at line 1:
ORA-20000: Price too high 
ORA-06512: at "MBAX4SD3.CHECK_ITEM_VALUE", line 13 
ORA-04088: error during execution of trigger 'MBAX4SD3.CHECK_ITEM_VALUE' 


SQL> INSERT into itemType VALUES ('D1', 'The other thing', '++-+-', 43.97, 'P');
INSERT into itemType VALUES ('D1', 'The other thing', '++-+-', 43.97, 'P')
            *
ERROR at line 1:
ORA-20000: Price too high 
ORA-06512: at "MBAX4SD3.CHECK_ITEM_VALUE", line 13 
ORA-04088: error during execution of trigger 'MBAX4SD3.CHECK_ITEM_VALUE' 


SQL> INSERT into itemType VALUES ('D2', 'The 3rd thing', '++-+-', 60.99, 'P');
INSERT into itemType VALUES ('D2', 'The 3rd thing', '++-+-', 60.99, 'P')
            *
ERROR at line 1:
ORA-20000: Price too high 
ORA-06512: at "MBAX4SD3.CHECK_ITEM_VALUE", line 13 
ORA-04088: error during execution of trigger 'MBAX4SD3.CHECK_ITEM_VALUE' 


SQL> INSERT into itemType VALUES ('D3', 'The 4th thing', '++-+-', 90.88, 'P');
INSERT into itemType VALUES ('D3', 'The 4th thing', '++-+-', 90.88, 'P')
            *
ERROR at line 1:
ORA-20000: Price too high 
ORA-06512: at "MBAX4SD3.CHECK_ITEM_VALUE", line 13 
ORA-04088: error during execution of trigger 'MBAX4SD3.CHECK_ITEM_VALUE' 


SQL> 
SQL> -- State after insertions
SQL> SELECT * FROM itemType;

ITEMNUM    NAME                 PICTURE      PRICE BELONGSTO                    
---------- -------------------- ------- ---------- ----------                   
A0         The Anarchy of Mask  ***          10.99 P                            
A1         The Butler Did       ---          11.99 C                            
A2         The Abolished Man    ===          15.99 SF                           
A3         Lyrical Bullets      +-+-         20.99 P                            
B1         The Postlude         =\=\=        34.99 P                            
C1         The August of Guns   **--         10.99 H                            
C2         The Expectant Mirror ^-^          12.99 H                            
D4         The 5th thing        ++-+-        10.88 P                            

8 rows selected.

SQL> 
SQL> -- Valid update
SQL> UPDATE itemType SET price=10.15 WHERE itemNum='A1';

1 row updated.

SQL> 
SQL> -- Invalid updates
SQL> UPDATE itemType SET price=61.56  WHERE itemNum='A1';
UPDATE itemType SET price=61.56  WHERE itemNum='A1'
       *
ERROR at line 1:
ORA-20000: Price too high 
ORA-06512: at "MBAX4SD3.CHECK_ITEM_VALUE", line 13 
ORA-04088: error during execution of trigger 'MBAX4SD3.CHECK_ITEM_VALUE' 


SQL> UPDATE itemType SET price=86.56  WHERE itemNum='A2';
UPDATE itemType SET price=86.56  WHERE itemNum='A2'
       *
ERROR at line 1:
ORA-20000: Price too high 
ORA-06512: at "MBAX4SD3.CHECK_ITEM_VALUE", line 13 
ORA-04088: error during execution of trigger 'MBAX4SD3.CHECK_ITEM_VALUE' 


SQL> UPDATE itemType SET price=150.23 WHERE itemNum='A3';
UPDATE itemType SET price=150.23 WHERE itemNum='A3'
       *
ERROR at line 1:
ORA-20000: Price too high 
ORA-06512: at "MBAX4SD3.CHECK_ITEM_VALUE", line 13 
ORA-04088: error during execution of trigger 'MBAX4SD3.CHECK_ITEM_VALUE' 


SQL> 
SQL> -- State after updating
SQL> SELECT * FROM itemType;

ITEMNUM    NAME                 PICTURE      PRICE BELONGSTO                    
---------- -------------------- ------- ---------- ----------                   
A0         The Anarchy of Mask  ***          10.99 P                            
A1         The Butler Did       ---          10.15 C                            
A2         The Abolished Man    ===          15.99 SF                           
A3         Lyrical Bullets      +-+-         20.99 P                            
B1         The Postlude         =\=\=        34.99 P                            
C1         The August of Guns   **--         10.99 H                            
C2         The Expectant Mirror ^-^          12.99 H                            
D4         The 5th thing        ++-+-        10.88 P                            

8 rows selected.

SQL> 
SQL> 
SQL> 
SQL> SPOOL OFF

\end{lstlisting}
\end{document}

